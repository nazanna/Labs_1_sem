\documentclass[a4paper,12pt]{article} % тип документа

% report, book

% Рисунки
\usepackage{graphicx}
\usepackage{wrapfig}
\usepackage{mathtext}
\usepackage[left=2cm,right=2cm,
    top=2cm,bottom=2cm,bindingoffset=0cm]{geometry}

\usepackage{hyperref}
\usepackage[rgb]{xcolor}
\hypersetup{				% Гиперссылки
    colorlinks=true,       	% false: ссылки в рамках
	urlcolor=blue          % на URL
}

%  Русский язык
\usepackage[T2A]{fontenc}			% кодировка
\usepackage[utf8]{inputenc}			% кодировка исходного текста
\usepackage[english,russian]{babel}	% локализация и переносы
\addto\captionsrussian{\def\refname{Список используемой литературы}}


% Математика
\usepackage{amsmath,amsfonts,amssymb,amsthm,mathtools} 
\usepackage{titlesec}
\titlelabel{\thetitle.\quad}

\usepackage{wasysym}

\begin{document}\begin{titlepage}

\thispagestyle{empty}

\centerline{МОСКОВСКИЙ ФИЗИКО-ТЕХНИЧЕСКИЙ ИНСТИТУТ}
\centerline{(НАЦИОНАЛЬНЫЙ ИССЛЕДОВАТЕЛЬСКИЙ УНИВЕРСИТЕТ)}

\vfill

\centerline{\huge{Лабораторная работа 3.6.1}}
\centerline{\LARGE{<<Спектральный анализ электрических сигналов>>}}

\vfill

Студент группы Б02-109 \hfill Назарчук Анна

\vfill

\centerline{Долгопрудный, 2022}
\clearpage
\end{titlepage} 
\section{Аннотация}
В работе исследованы спектры периодических сигналов: модулированный по амплитуде, прямоугольные импульсы и цуги. Проверены теоретические зависимости параметров спектра и соотношения неопределенности на практике .


\section{Введение}
Многие практические задачи описания поведения некоторой системы во времени зачастую сводятся к выяснению связи между сигналом, подаваемым на <<вход>>
системы (обозначим его как $f(t)$), и её реакцией на <<выходе>> $g(t)$). Суть спектрального метода состоит в представлении произвольного воздействия в виде суперпозиции откликов на некоторые элементарные слагаемые. Данный метод используется для анализа многих сигналов, поэтому необходимо экспериментально ознакомиться с ним, сгенерировать и получить на осциллографе спектры различных периодических сигналов, проверить экспериментально параметры спектра и некоторые теоретические соотношения между ними.

\section{Методика измерений}
Основным соотношением для спектра является соотношение неопределенностей, верное для любого сигнала \cite{labnik}:
\begin{equation}
\Delta \omega \cdot \Delta t \sim 2\pi
\end{equation}

\begin{figure}[h!]
\begin{center}
\includegraphics[width=\textwidth]{пример}
\caption{Примеры сигналов а) периодической последовательности прямоугольных импульсов, б) периодической последовательности цуг, в) модулированного по амплитуде сигнала из \cite{labnik}} \label{пример}
\end{center}
\end{figure}
Рассмотреть все сигналы невозможно, поэтому работа проводилась на трех относительно простых сигналах. Для каждого из них сгенерирован сигнал определенной формы, обработан с помощью цифрового осциллографа, проверены соотношения неопределенности с помощью курсорных измерений.

1. Первая часть работы заключалась в исследовании спектра периодической последовательности прямоугольных импульсов (пример показан на рисунке \ref{пример}). 
Теоретически рассчитано значение коэффициентов $c_n$ \cite{labnik}, которое проверено экспериментально:
\begin{equation}
c_n  = \dfrac{sin(\pi n \tau / T))}{\pi n}
\end{equation}

2. Вторая часть работы состояла в исследовании спектра периодической последовательности цугов гармонических колебаний (пример показан на рисунке \ref{пример}).
Теоретически известен спектр сигнала \cite{labnik}: 
\begin{equation}
F(\omega) = \dfrac{\tau}{2T}\left[\dfrac{\sin(\omega-\omega_0)\tau /2}
{(\omega-\omega_0)\tau /2}
 + \dfrac{\sin(\omega+\omega_0)\tau /2}{(\omega+\omega_0)\tau /2}\right]
\end{equation}

3. Последняя часть заключалась в исследовании спектра гармонических сигналов, модулированных по амплитуде (пример показан на рисунке \ref{пример}).
Теоретический вид сигнала \cite{labnik}: 
\begin{equation}
f(t) = a_0 \cos (\omega_0 t) +\dfrac{ma_0}{2}\cos (\omega_0 +\Omega)t+\dfrac{ma_0}{2}\cos (\omega_0 -\Omega)t
\end{equation}
где $m$ - глубина модуляции.
Модулированное колебание представляется в виде:
\begin{equation}
f(t)=f_0(t)+f_1(t)+f_2(t),\hspace{3mm} f_0(t)=a_0 \cos (\omega_0 t),\hspace{3mm} f_1(t)=\dfrac{ma_0}{2}\cos (\omega_0 +\Omega)t,\hspace{3mm} f_2(t)=\dfrac{ma_0}{2}\cos (\omega_0 -\Omega)t
\end{equation}
$f_0$ - основная гармоника, $f_1, f_2$ - боковые гармоники
Для модулированного по амплитуде сигнала существует теоретическое соотношение между амплитудами гармоник, которое можно проверить экспериментально:
\begin{equation}
\dfrac{a_{бок}}{а_{осн}}=\frac{1}{2}
\end{equation} 


\section{Результаты и их обсуждение}
\subsection*{Исследования спектра периодической последовательности прямоугольных импульсов}
Для исследования периодической последовательности прямоугольных импульсов на генераторе создан сигнал с разными параметрами, по которому на экране осциллографа получен спектр (рис. \ref{прямоуг})

\begin{figure}[h!]
\begin{minipage}[h!]{0.47\linewidth}
\center{\includegraphics[width=1\linewidth]{1_1}} a) $\nu_{повт} = 1000 Гц, \tau = 50 мкс$\\
\end{minipage}
\hfill
\begin{minipage}[h!]{0.47\linewidth}
\center{\includegraphics[width=1\linewidth]{1_2}} \\b) $\nu_{повт} = 1400 Гц, \tau = 50 мкс$
\end{minipage}
\end{figure}
\begin{figure}[h!]
\begin{minipage}[h!]{0.47\linewidth}
\center{\includegraphics[width=1\linewidth]{1_3}} c) $\nu_{повт} = 700 Гц, \tau = 50 мкс$ \\
\end{minipage}
\hfill
\begin{minipage}[h!]{0.47\linewidth}
\center{\includegraphics[width=1\linewidth]{1_4}} d) $\nu_{повт} = 1000 Гц, \tau = 70 мкс$ \\
\end{minipage}
\caption{Спектры последовательностей прямоугольных импульсов при разных частотах повторения и длительности импульса}
\label{прямоуг}
\end{figure}
\newpage
При $\nu_{повт} = 700 Гц$ проведены измерения ширины спектра. Результаты 
представлены на рисунке \ref{dnu(tau)_img}.

\begin{figure}[h!]
\begin{center}
\includegraphics[width=\textwidth]{dnu(tau)}
\caption{Зависимость ширины спектра от длительности спектра для последовательности прямоугольных импульсов при частоте повторения $\nu_{повт} = 700 Гц$} \label{dnu(tau)_img}
\end{center}
\end{figure}
\begin{figure}[h!]
\begin{center}
\includegraphics[width=\textwidth]{a(n)}
\caption{Теоретический спектр прямоугольных импульсов при частоте повторения $\nu_{повт} = 1000 Гц$ и длительности импульса $\tau = 50 мкс$ из \cite{labnik}} \label{теор}
\end{center}
\end{figure}
Рассчитан коэффициент наклона прямой:
\begin{equation}
k = 0.9997 \pm 0.0039
\end{equation}
Полученное значение близко к $1$, что подтверждает соотношение неопределенностей. 


Для сравнения экспериментальных и теоретических значений спектра для одного из сигналов (a) с рис. \ref{прямоуг}) рассчитана теоретическую зависимость и изображена на графике \ref{теор}. Теоретический и экспериментальный спектр похожи, что показывает справедливость теоретического расчета в данном случае.


\newpage
\newpage
\newpage
\subsection*{Исследование спектра периодической последовательности цугов гармонических колебаний}

\begin{figure}[h!]
\begin{minipage}[h!]{0.47\linewidth}
\center{\includegraphics[width=1\linewidth]{2_1}} a) $\nu = 50 кГц, T = 1 мс, N = 5$\\
\end{minipage}
\hfill
\begin{minipage}[h!]{0.47\linewidth}
\center{\includegraphics[width=1\linewidth]{2_2}} \\b) $\nu = 50 кГц, T = 1 мс, N = 3$
\end{minipage}
\vfill
\begin{minipage}[h!]{0.47\linewidth}
\center{\includegraphics[width=1\linewidth]{2_3}} c) $\nu = 50 кГц, T = 3 мс, N = 5$ \\
\end{minipage}
\hfill
\begin{minipage}[h!]{0.47\linewidth}
\center{\includegraphics[width=1\linewidth]{2_4}} d) $\nu = 30 кГц, T = 1 мс, N = 5$ \\
\end{minipage}
\vfill
\begin{minipage}[h!]{0.47\linewidth}
\center{\includegraphics[width=1\linewidth]{2_5}} \\e) $\nu = 70 кГц, T = 1 мс, N = 5$
\end{minipage}
\caption{Вид спектра для периодической последовательности цугов при разных частотах несущей $\nu$ = 50 кГц, периодах повторения $T$ = 1 мс, числах
периодов в одном импульсе $N$ = 5}
\label{спектр_цуги}
\end{figure}
Для исследования спектра периодической последовательности цугов гармонических колебаний на генераторе создан сигнал последовательности синусоидальных цугов с разными параметрами, по которому на экране осциллографа получен спектр. (рис. \ref{спектр_цуги})

Для проверки соотношения неопределенностей для данного сигнала при фиксированной длительности импульсов $\tau$ = 50 мкс измерены расстояния между соседними спектральными компонентами от периода повторения импульсов (рис. \ref{dnu(T)_img})

\begin{figure}[h!]
\begin{center}
\includegraphics[width=0.9\textwidth]{T(dnu)}
\caption{Зависимость расстояния между соседними спектральными компонентами от периода повторения импульсов для периодической последовательности цугов при часоте несущей $\nu$ = 50 кГц и числе
периодов в одном импульсе $N$ = 5} \label{dnu(T)_img}
\end{center}
\end{figure}

Теоретически известно (\cite{labnik}), что точки должны хорошо ложиться на прямую, однако из графика видно, что это не так. Проблема заключается в снятии данных (был выбран неверный канал при курсорных измерениях). Поэтому подтвердить справедливость соотношения неопределенности по данным экспериментальным значениям невозможно. 
\subsection*{Исследование спектра гармонических сигналов, модулированных по амплитуде}
Для исследования спектра гармонических сигналов, модулированных по амплитуде на генераторе создан сигнал, модулированный по амплитуде, по которому на экране осциллографа получается спектр (\ref{мод}).
\begin{figure}[h!]
\begin{center}
\includegraphics[width=0.75\textwidth]{3}
\caption{Спектр сигнала, модулированного по амплитуде, при частоте несущей $\nu_0$ = 50 кГц, частоте модуляции $\nu_{мод}$ = 2 кГц} \label{мод}
\end{center}
\end{figure}
Измерена с помощью осциллографа глубина модуляции $m$:
\begin{equation}
m = \dfrac{A_{max}-A_{min}}{A_{max}+A_{min}} = \dfrac{1.54 - 0.54}{1.54 + 0.54} = 0.5, что \hspace*{1mm} сходится \hspace*{1mm}с\hspace*{1mm} установленным \hspace*{1mm}на\hspace*{1mm} генераторе
\end{equation}
Для проверки теоретической зависимости, изменяя глубину модуляции, измерена $\dfrac{a_{бок}}{а_{осн}}$ - отношение амплитуд боковой и центральной полос спектра (рис. \ref{mod_img}).

\begin{figure}[h!]
\begin{center}
\includegraphics[width=0.75\textwidth]{a(m)}
\caption{Зависимость $\dfrac{a_{бок}}{а_{осн}}$ от $m$ для сигнала, модулированного по амплитуде, при частоте несущей $\nu_0$ = 50 кГц, частоте модуляции $\nu_{мод}$ = 2 кГц} \label{mod_img}
\end{center}
\end{figure}
Определен коэффициент наклона прямой:
\begin{equation}
k = 0.502 \pm 0.002
\end{equation} 
Результат сходится с предсказанным теоретически (0.5).

\section{Выводы}


\hspace{4mm} 1. При исследовании последовательности прямоугольных импульсов получена зависимость ширины спектра от длительности импульса, что подтверждает соотношение неопределенностей для данного вида сигнала: $\tau \cdot \Delta\nu \sim 1$.

2. Проверены теоретические расчеты спектра при прямоугольных импульсах (теоретическая и экспериментальная картины схожи).

3. При обработке данных от спектра периодической последовательности цугов была обнаружена ошибка при снятии данных, что не позволило проверить соотношение неопределенностей.

4. Получен угол наклона графика зависимости $\dfrac{a_{бок}}{а_{осн}}$ от $m$ ($k$=0.5), подтверждено теоретическое значение этого угла ($k$=0.5).


\begin{thebibliography}{}
    \bibitem{labnik}  Никулин М.Г., Попов П.В., Нозик А.А. и др. Лабораторный практикум по общей физике : учеб. пособие. В трех томах. Т. 2. Электричество и магнетизм
\end{thebibliography}


\end{document}